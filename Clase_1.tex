\documentclass[utf8x]{beamer}
\usepackage[spanish]{babel}
\usepackage{palatino}
\usepackage{amsmath}
\usepackage{latexsym}
\usepackage{amssymb}
\usepackage{amsfonts}
\usepackage{fancybox}
\newcommand{\R}{\mathbb{R}}
\newcommand{\N}{\mathbb{N}}
\newcommand{\M}{\mathcal{M}}
 \usetheme{Warsaw}


\title{La Diferencial}
\author{Verónica Brice\~no V.}
\date{septiembre 2018}

\begin{document}

\begin{frame}
\frametitle{La Diferencial}
\titlepage
\end{frame}


%%%%%%%%%%%%%%%%

\begin{frame}
\frametitle{Linealización}

\begin{block}{Recordar...}

$$f'(a) = \lim_{x\rightarrow a} \frac{f(x) - f(a)}{x-a}$$

\end{block}


Notar que:

Si $x$ está cerca de $a$, se tiene:
$f'(a)  \sim \frac{f(x) - f(a)}{x-a}$


Entonces,
$f(x)  \sim  f(a)  + f'(a)(x-a)$



\begin{block}{Definición}

Sea $P=(a,f(a))$. La recta tangente a la gráfica de $f$ en el punto $P$, está dado por:


$p(x) =  f(a)  + f'(a)(x-a)$

Se llama LINEALIZACIóN de $f$ en $a$.


Representa una aproximación de $f$ cerca de  $a$.


$p$ es un polinomio de grado 1.

\end{block}

\end{frame}


\begin{frame}
\frametitle{Geométricamente....}

\begin{center}
\includegraphics[width=1\textwidth]{Graf_Diferencial1.png}

\end{center}

\end{frame}



%%%%%%%%%%%%%%%%

\begin{frame}
\frametitle{Linealización}

Considerar un pequeño incremento: $\Delta x$.
Entonces:

$p(a+\Delta x) = f(a) + f'(a)(a + \Delta x - a) = f(a)+f'(a)\Delta x $

Se define: $\Delta f := f(a+\Delta x) - f(a)$


Por tanto, $ f(a+\Delta x)  = f(a) +\Delta f$

\end{frame}

%%%%%%%%%%%%%%%%


\begin{frame}{Ejemplo:}

Calcular $f(a+\Delta x)$ y $p(a+\Delta x)$ cuando:

$f(x) = x^3$

$a= 1$

$\Delta x = 0,1$

Tenemos:

$a+\Delta x = 1,1$

$f(1,1)=(1,1)^3=1,331$

Linealización:
$p(1,1)=f(1)+f'(1)\cdot 0,1=1+3\cdot 0,1=1,3$

\end{frame}

\begin{frame}{Observación}

\begin{block}{Notar que:}

 $f(a+\Delta x)$ aproxima (se parece al valor de) a  $p(a+\Delta x)$ si y solo si 
 $\Delta f$ y $f'(a)\Delta x $ son parecidas.

\end{block}

Esto nos da la idea de definir...
\end{frame}

%%%%%%%%%%%%%%%%

\begin{frame}{Diferencial}

\begin{block}{Definición}


Se llama DIFERENCIAL de $f$ a la parte principal de su incremento lineal respecto al incremento $\Delta x = dx $ de la variable independiente $x$.


La diferencial de una función es igual al producto de su derivada por la diferencial de la variable independiente.


$$ dy = y' dx $$


Esto implica:

$$  y'   = \frac{dy}{dx} $$


\end{block}

\end{frame}

%%%%%%%%%%%%%%%

\begin{frame}{Geometricamente....}

\begin{center}
\includegraphics[width=1\textwidth]{Graf_Diferencial2.png}
\end{center}


\end{frame}


%%%%%%%%%%%%%%%


\begin{frame}{Observación}

\begin{block}{Notar que:}
\begin{itemize}
\item
 $dy =df$

\item
Es comúun escribir $x$ en vez  de $a$. Por eso escribimos:

$df = f'(x)\Delta x$
\end{itemize}

\end{block}

\end{frame}



\begin{frame}
\frametitle{Ejemplo 1:}

Hallar el incremento y la diferencial de la función $y= 5x+x^2$
para $a=2; \Delta 0,001$

Se tiene: $dy=(5+2x)dx$

Evaluamos: $ fy=(5+4)0,001=0,009$

Ahora: $a+\Delta x=2.001$

Incremento: $\Delta y= (5\cdot 2\cdot 0,001+(2\cdot 0.001)^2)
-(5\cdot2 +4=0,009001)$


\end{frame}


\begin{frame}
\frametitle{Ejemplos:}

\begin{enumerate}
\item
Calcular, en el ejemplo anterior,  para $a= 1, \Delta x = 0,01$

\item
En cuánto aumentará aproximadamente el lado de un cuadrado si su área aumenta de 9 $m^2$ a 9,1 $m^2$?

\item
Hallar el valor aproximado de $\sen 31^o$.


\end{enumerate}

\end{frame}

%%%%%%%%%%%%%

\begin{frame}
\frametitle{Ejercicios:}

\begin{enumerate}
\item
Hallar el incremento y la diferencial de la función $y= 5x+x^2$ para $a= 2, \Delta x = 0,001$.
\item
Encontrar la linealización de $f(x) = \tg x $ en $x= \frac{\pi}{4}$

\item
Aproximar el valor de $\sqrt[3]{29}$

\item
Demostrar que cualquiera que sea $x$ el incremento de la función $y=2^x$ es equivalente a $2^x \Delta x \ln 2$

\end{enumerate}

\end{frame}


%%%%%%%%%%%%%

\begin{frame}
\frametitle{Error}

\begin{block}{Definición}
En general, si se puede precisar que al aproximar una cantidad Q se tiene un error E,   podemos calcular el ERROR RELATIVO, como:  $$\frac{E}{Q}$$


Este error es generalmente expresado en forma de porcentaje y mide que tan grande es el error comparado con la cantidad que se esta midiendo.

\end{block}

\end{frame}

%%%%%%%%%%%%%%%

\begin{frame}
\frametitle{Ejercicios:}

\begin{enumerate}
\item
El error al medir el lado de un cubo es a lo sumo de 1 por ciento.
 ¿Qué   porcentaje de error se obtiene al estimar el
volumen de un cubo?

Desarrollo:


$x$: longitud del lado

$\delta x$: error al aproximar $x$

$V=x^3$: volumen

$dV=3x^2 dx$

$$\vert \frac{dx}{x}\vert < 1$$

$dV$: error en el  volumen

$$\vert\frac{dV}{V}\vert =\vert\frac{3x^2dx}{x^3}\vert=
3\vert\frac{dx}{x}\vert < 3\cdot 0.01=0.03$$

Respuesta: error relativo  es menor al  $3\%$.
	
\item
Demostrar que un error relativo de un $1\%$ cometido al determinar el radio da lugar a un error relativo aproximado de un $2\%$ al calcular el área de la  circunferencia y la superficie de la esfera.

\end{enumerate}

\end{frame}



%%%%%%%%%%%

\begin{frame}
\frametitle{Teorema}

Sean $f$ y $g$ funciones diferenciables. Entonces:

\begin{enumerate}
\item
$d(f+g) = df + dg$

\item
$d(\alpha f) =   \alpha df$

\item
$ d(f \cdot g) = (df)g + f(dg)$

\item
$ d(\frac{f}{g}) = \frac{(df)g - f(dg) }{g^2}$

\item
$ d(f \circ g) = f'(g) dg  $

\item
$d (c)  = 0$ ,  donde $c$ es una constante.
\end{enumerate}


Observación:

Resultan muy útiles estas propiedades cuando buscamos la diferencial de una función definida implicitamente.

\end{frame}



\begin{frame}
\frametitle{Ejercicios:}

Hallar $dy$ si:

\begin{enumerate}
\item
 $x^2+2xy-y^2 = a^2$, donde $a \in \R$

\item
$y=e^{\frac{-x}{y}}$

\item
$\ln (\sqrt{x^2+y^2}) = \arctg (\frac{y}{x})$

\item
$(x+y)^2(2y+x)^3  = 1$

\end{enumerate}

\end{frame}


\begin{frame}
\frametitle{Ejercicios Propuestos:}
\begin{itemize}
\item
Demostrar basándose en la fórmula de la ley de Ohm 

$I=\frac{E}{R}$ 
que una peque\~na variación de la intensidad de la corriente, debida a una peque\~na resistencia, puede hallarse de forma aproximada por la fórmula:
$\Delta I = - \frac{I}{R} \Delta R$

\item
Usar diferenciales para aproximar $\sqrt{99,4}$.

\item
El beneficio de una empresa, está dada por:
$B(x) = (500x-x^2) - (\frac{1}{2}x^2  - 77x  + 300)$
donde x: cantidad de unidades vendidas.
Aproximar el cambio y el porcentaje de cambio de los beneficios si la empresa pasa de producir 115 a 120 unidades.
\end{itemize}
\end{frame}


\begin{frame}
\frametitle{ Ejercicios Propuestos:}
\begin{itemize}
\item
Un ingeniero se encuentra al nivel de la base de un edificio, a una distancia de 30 metros de éste,  mide el ángulo de elevación a la parte superior del edificio y éste es de 75 grados. >Cuál es el máximo error con que se debe medir el ángulo para que el porcentaje de error en la estimación de la altura del edificio sea menor del 4\%? (Control 1 - II Sem 2012)

\item
Si al medir el radio de una esfera da 6 cm con un posible error de 
$\pm 0,02$ cm, utilizando diferenciales, estime el máximo error
 posible en calcular el volumen de la esfera.
 (Control 1 - I Sem 2012)
\end{itemize}
\end{frame}


\begin{frame}
\frametitle{ Ejercicios Propuestos:}
\begin{itemize}
\item
Un cono tiene altura igual al radio basal. Se mide la altura de este cono, encontrándose que es de 10 cm con un error de 0,5$\%$.
Aproxime el error relativo del volumen por la diferencial.

(Certamen 1 - II Semestre 2015)


\end{itemize}
\end{frame}

\end{document}






