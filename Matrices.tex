\title{Matrices}
\author{Verónica Brice\~no V.}
\date{Septiembre 2018}

\begin{document}

En este documento veremos:

\begin{itemize}
\item
Definición de matriz
\item
Definiciones  de matrices especiales
\item
Operatoria con matrices
\item
Ejemplos
\item
Operaciones Elementales
\item
Rango
\item
Sistemas de Ecuaciones
\item
Método de Eliminación Gaussiana
\item
Método de Eliminación Gauss - Jordan
\item
Matriz Inversa
\item
Determinante
\item
Inversa por menores
\item
Regla de Cramer.
\end{itemize}



{Sistemas de Ecuaciones Líneales}

\begin{itemize}
\item
La principal aplicación que tienen las matrices se presenta en los sistemas de ecuaciones líneales.

\item
Ejemplo:

\begin{eqnarray*}
5x- 2y +z  &=& 0\\
3x+4z       &=&  0
\end{eqnarray*}
\end{itemize}




{Matrices - Definición}

\begin{itemize}
\item
Dados $n, m \in  \N$, se define una matriz de orden $n \times m$, como un arreglo de números dispuestos en  $n$ filas y $m$ columnas, se escribe: 

\item
$A = \left(
\begin{array}{cccc}
a_{11} & a_{12} & \cdots & a_{1m}\\
a_{21} & a_{22} & \cdots & a_{2m}\\
a_{31} & a_{32} & \cdots & a_{3m}\\
\vdots  & \vdots  & \cdots & \vdots\\
a_{n1} & a_{n2} & \cdots & a_{nm}
\end{array}
\right)$
\end{itemize}


{Matrices - Definición}

\begin{itemize}
  \item
  En forma reducida escribimos: $(a_{ij})_{n\times m}$, también es 
  común denotar las matrices con letras mayúsculas como: A, B, C, etc.
  \item
  Notar que  $a_{ij}$ representa el elemento de la matriz $A$ ubicado en la fila $i$ y la columna $j$.
  \item
  Los elementos $a_{ij}$ (también llamados coeficientes) son números que pueden pertenecer a cualquier conjunto númerico, como por ejemplo $\R$ o  $\Co$.
  \item
  Así, denotamos,  $\M_{n\times m}(\mathbb{R})$ o  $\M(n\times m, \R)$, al conjunto de todas las matrices de orden $n \times m$, cuyos coeficientes pertenecen a $\R$.
  \item
  Analogamente se define:  $\M_{n\times m}(\Co)$ o  $\M(n\times m, \Co)$.
\end{itemize}


{Ejemplos}
\begin{enumerate}
\item
$ A = \left(
\begin{array}{ccc}
1 & -8 & 0\\
2 & 0 & 3
\end{array}
\right)
$

 de orden $2 \times 3$

coeficientes reales.


Por tanto, $A \in \M_ {2 \times 3}(\R)$.


\item
$ B = \left(
\begin{array}{cc}
-1 & 0\\
0  & i\\
5 & 0
\end{array}
\right)
$
 de orden $3 \times 2$

coeficientes complejos.

Por tanto, $B \in \M_ {3 \times 2}(\Co)$.

\item
$C = \left(
\begin{array}{cccc}
1. 234 & -5 & x & 0.234\\
280 & 0 & x^2+2 & 1.22
\end{array}
\right)
$
 de orden $2 \times 4$.

El conjunto al que pertenece  $C$ dependerá del valor de $x$.
\end{enumerate}


{Matriz Columna / Matriz Fila}

Una matriz de orden $n\times 1$, se llama matriz columna o vector columna.

Tiene la forma:

$\left(
\begin{array}{c}
a_{11}\\
a_{21}\\
a_{31}\\
\vdots \\
a_{n1}
\end{array}
\right)$

Una matriz de orden $1\times m$, se llama matriz fila o vector fila.

Tiene la forma:

$\left( a_{11}  a_{12}  \cdots  a_{1m} \right )$


{Matriz Nula}

La matriz $A=(a_{ij})_{n\times m}$, tal que $a_{ij}=0, \forall i= 1, \cdots, n$, y $ \forall j= 1, \cdots, m$,
 se llama Matriz Nula de orden $n\times m$, se denota $A=(0)_{n\times m}$.  Esto es,

$(0)_{n\times m} = \left(
\begin{array}{cccc}
0 & 0 & \cdots & 0\\
0 & 0 & \cdots & 0\\
0 & 0 & \cdots & 0\\
\vdots  & \vdots  & \cdots & \vdots\\
0 & 0 & \cdots & 0
\end{array}
\right)$


{Matrices Cuadradas}

\begin{itemize}
\item
Una  matriz de orden $n\times n$ esto es, igual número de filas y columnas,  se dice matriz cuadrada
de orden $n$.

\item
Sea  $A=(a_{ij})_{n\times n}$, una matriz cuadrada. 

\item
Los coeficientes $a_{ii}$, para $i=1, \cdots, n$, forman la Diagonal Principal de la matriz A.

\item
Los coeficientes $a_{i,n+1-i}$, para $i=1, \cdots, n$, forman la Diagonal Secundaria de la matriz A.
\end{itemize}


{Diagonal Principal y Secundaria}

Diagonal Principal:
$   \left(
\begin{array}{cccc}
a_{11} &             &             &     *    \\
            & a_{22} &             &            \\
            &             & \ddots  &             \\
     *     &            &               & a_{n n}
\end{array}
\right)$


Diagonal Secundaria:

$  \left(
\begin{array}{cccc}
  *             &              &                 &   a_{1n}      \\
                 &              & a_{2,n-1}  &            \\
                 & \ldots   &                 &            \\
a_{n1}      &              &                 & *
\end{array}
\right)$



{Matriz Diagonal}

\begin{itemize}
\item
Una matriz cuadrada de orden $n$, se dice Matriz Diagonal si todos los elementos fuera de la diagonal principal son nulos.

\item
IMPORTANTE:
Los elementos de la diagonal no necesariamente son distintos de cero.

\item
Una matriz diagonal de  orden $n$, se llama Matriz Identidad, si todos los elementos de la Diagonal Principal son iguales a 1.

\item
Notación: $I_n$.
\end{itemize}


{Matriz Identidad}

Esto es, 
$I_n =   \left(
\begin{array}{cccc}
1           &   0        &  \cdots    &     0\\
0           &   1        &                &     0 \\
\vdots   &   0           & \ddots  &      0 \\
0           &       \cdots      &       & 1
\end{array}
\right)$


Veamos algunos ejemplos en la pizarra... $I_2, I_3...$


{Matriz Triangular Inferior y Superior}

\begin{itemize}
\item
Sea $A = (a_{ij})$ una matriz cuadrada de orden $n$.

\item
Si se verifica que todos los elementos sobre su diagonal
principal son ceros (no importan los demás) se dice  que $A$ es una Matriz Triangular Inferior.

\item
Si se verifica que todos los elementos bajo su diagonal
principal son ceros (no importan los demás) se dice  que $A$ es una Matriz Triangular Superior.

\item
Entonces, ¿qué forma tienen estas matrices?
\end{itemize}


{Matriz Triangular Inferior y Superior}

Matriz Triangular Inferior: $ \left(
\begin{array}{cccc}
    *       &     0     &      0      &   0    \\
    *       &     *     &     0      &    0      \\
    *      &      *    & \ddots  &     0      \\
    *      &      *    &       *        & *
\end{array}
\right)$


Matriz Triangular Superior: $   \left(
\begin{array}{cccc}
*           &    *        &      *      &     *    \\
    0       & *           &      *      &    *      \\
    0      &       0     & \ddots  &       *      \\
    0      &        0    &       0        & *
\end{array}
\right)$

Analíticamente, ¿cómo podemos definir estas matrices?


Matriz Triangular Inferior: $a_{ij}=0$ si $i<j$.


Matriz Triangular Superior: $a_{ij}=0$ si $i>j$.


{Algebra}

\begin{itemize}
\item
\textit{Igualdad de Matrices:}

Sean $A=(a_{ij})$ y $B=(b_{ij})$ dos matrices.

Se dice que $A=B$, si las matrices son del mismo orden, y además $a_{ij} = b_{ij}, \forall i,j $.

\item
\textit{Suma de Matrices:}

Si $A=(a_{ij})$ y $B=(b_{ij}) $ son dos matrices de orden $n\times m$.

Se define $A + B =(a_{ij} + b_{ij}) $, de orden  $n\times m$.

\item
Multiplicación por escalar:

Si $A=(a_{ij})_{n\times m}$ y $\alpha \in \mathbb{R}$ o  $\mathbb{C}$, entonces $\alpha A= (\alpha a_{ij})_{n\times m}$.

\item
Multiplicación de dos matrices:

Sea $A=(a_{ij})_{n\times m}$ y $B=(b_{ij})_{m \times p}$. 

Entonces se define la matriz $C$ de orden $n\times p$, como $C= A \cdot B$, donde los coeficientes $c_{ij}$ se obtienen:

$$c_{ij}= \sum_{k=1}^{m}a_{ik}\cdot b_{kj}$$

\end{itemize}


{Ejemplos Operatoria}

\begin{itemize}
\item
$
 \left(
\begin{array}{ccc}
1 & -5 & b\\
2 & 0 & x^2+2
\end{array}
\right)
=
 \left(
\begin{array}{ccc}
1 & -5 & 3 \\
2 & a &  11
\end{array}
\right)
$


Entonces, $a= 0$, $b=3$, y $x=3$ ó  $x=-3$

\item
 Sea $A= 
\left(
\begin{array}{cc}
5 & -1 \\
2 & 0 \\
0 & 4
\end{array}
\right)
$
 y
$B=  \left(
\begin{array}{cc}
1  & 3 \\
1  & -7\\
-1 & 0
\end{array}
\right)
$


Entonces, $A+B=
\left(
\begin{array}{cc}
6  & 2 \\
3  & -7\\
-1 & 4
\end{array}
\right)
$


\item
Sea $A=
\left(
\begin{array}{ccc}
2 &  0  & 2 \\
-1 & 1 & \frac{3}{4}
\end{array}
\right)
$
y
$\alpha = -4$.


Entonces, $\alpha \cdot A=
\left(
\begin{array}{ccc}
-8 &  0  & -8 \\
4  & -4  & -3
\end{array}
\right)
$
\end{itemize}


{Ejemplos Operatoria}

Sea $A=
\left(
\begin{array}{cc}
2 &  1  \\
0 & -3 \\
1 & -1
\end{array}
\right)
$
y
$B=  \left(
\begin{array}{ccc}
0  & 1  & 1\\
-1 & 0  & -1
\end{array}
\right)
$


Entonces, $A \cdot B=$

$\left(
\begin{array}{ccc}
2\cdot 0 + 1\cdot (-1)       &   2\cdot 1+ 1\cdot 0        & 2\cdot 1 + 1\cdot (-1)     \\
0\cdot 0 + (-3)\cdot (-1)   &   0\cdot 1+ (-3)\cdot 0    & 0\cdot 1 + (-3)\cdot (-1)  \\
1\cdot 0 + (-1)\cdot (-1)   &   1\cdot 1+ (-1)\cdot 0    & 1\cdot 1 + (-1)\cdot (-1)  \\
\end{array}
\right)
=
\left(
\begin{array}{ccc}
-1       &   2        & 1     \\
3       &   0        & 3     \\
1        &   1         & 2    \\
\end{array}
\right)
$


{Ejercicios Propuestos}

1.- Sea 

 $ A= \begin{bmatrix}
2 &  1 & -2 \\
-3 &  0 &  4
\end{bmatrix}$
, 
 $ B= \begin{bmatrix}
 0 &   1\\
-3 &  2  \\
 8 &  1
\end{bmatrix}$
, 
 $ C= \begin{bmatrix}
2   &  -2   & 4   \\
5   &  -6  &  7\\
2   &   1   & -3
\end{bmatrix}$
,
$ D= \begin{bmatrix}
2 &   1\\
-3 &  2  \\
 3  &  4
\end{bmatrix}$
y 
$ E= \begin{bmatrix}
2   &  -2   & 1   \\
3   &  5    &  6\\
-5  &   1   & 4
\end{bmatrix}$

Calcular, si es posible:  
$E+C$, $CB+D$, $AB+D^2$, $A(BD)$, $AC+AE$.


2.- Encontrar todas las matrices que conmutan con:
$ A= \begin{bmatrix}
1 & -2\\
-3 & 4
\end{bmatrix}$

3.- Si
$ A= \begin{bmatrix}
0 & i \\
i &  0 
\end{bmatrix}$
Calcular $A^2, A^3, \cdots, A^n$.
Qué es  $A^0$?

4.- Determinar $2 \cdot A^2 +A \cdot B$,  si $A = (i )_{3\times 3}$ y $B = (j )_{3\times 3}$.


Para mas ejercicios consultar www.aula.usm.cl


{Traza}

\begin{itemize}
\item
Se llama Traza de una matriz $A = (a_{ij})_{n \times n}$ a la suma de los elementos de la diagonal principal.

\item
Esto es, $tr(A) = \sum_{i=1}^n a_{ii}$

\item
Observación:

$tr(A+B) = tr(A) + tr(B)$
\end{itemize}


{Observaciones}

\begin{enumerate}
\item
Cómo se define la diferencia?

 $A-B=$
 
$A+(-B)$

\item
Cómo se define la división?
 
$$\frac{A}{B}=A \cdot B^{-1}$$

Veamos algunos ejemplos...

Entonces,  en algunos casos $A^{-1}$ no existe.
\end{enumerate}


{Propiedades de Operatoria de Matrices}

Sean $A, B$ y $C$ matrices (el orden de cada matriz se asume que es el apropiado para que cada operación tenga sentido).

\begin{enumerate}
\item
$A + B = B +A$

\item
$   (A + B)+C = A +(B +C) $

\item
$ A + (0) = A  $

\item
$  A +(-1)\cdot A =  (0)    $

\item
$\alpha \cdot(A + B) = \alpha \cdot A + \alpha \cdot B   $
 
\item
$ (\alpha+  \beta) \cdot A = \alpha \cdot A + \beta \cdot A  $

\item
$ (\alpha \cdot  \beta) \cdot A = \alpha ( \beta \cdot A )   $

\item
$ 1 \cdot A = A  $

\item
$  (A \cdot B) \cdot C = A \cdot (B \cdot C)  $

\item
$ A\cdot  (B +C) = A\cdot B +A\cdot C  $

\item
$ (A + B) \cdot  C = A\cdot C + B \cdot C  $

\item
$\alpha \cdot (A \cdot B) = (\alpha \cdot A)\cdot  B = A( \alpha \cdot B)  $

\item
$ A \in M_{n\times m} \Rightarrow I_n \cdot A = A = A \cdot I_m  $

\end{enumerate}


{Observación }

\begin{alertblock}{IMPORTANTE}
El producto matricial no  es conmutativo.
\end{alertblock}


Ejemplo:
Sea:
$A= \left(
\begin{array}{ccc}
1  & -1 \\
2  &   1
\end{array}
\right)
$
y
$B= \left(
\begin{array}{ccc}
1  & -1 \\
0  &   1
\end{array}
\right)
$


\begin{alertblock}{IMPORTANTE}
Para las matrices, no se verifica que: $A \cdot B =  0
 \Rightarrow A=0 \vee  B=0$.
\end{alertblock}


Ejemplo:

$ \left(
\begin{array}{ccc}
0  & 1 \\
0  & 0
\end{array}
\right)
\cdot
\left(
\begin{array}{ccc}
0  & 1 \\
0  & 0
\end{array}
\right)
=\left(
\begin{array}{ccc}
0  & 0 \\
0  & 0
\end{array}
\right)
$


{Observación }

Qué otras propiedades  NO se verifican en las matrices????

\begin{alertblock}{IMPORTANTE:}
$$(A+B)^2 \neq  A^2 + 2AB +B^2$$
\end{alertblock}

\begin{alertblock}{IMPORTANTE:}
$$A^2  - B^2 \neq  (A+B)(A-B)$$
\end{alertblock}


{Observación }

Tiene sentido las potencias de matrices???

Solo si $A$ es una matriz cuadrada.

Así, 

$$A^k =
 \left\{ \begin{array}{lcl}
I_n & \mbox{si} & k=0 \\ 
A & \mbox{si} & k=1 \\ 
A\cdot A  \cdot A ... A & \mbox{si} & k>1 (k \mbox{  veces}) 
\end{array}\right.
$$

y si $ k<0$???


Veremos esto mas adelante...


{Definición}

\begin{itemize}
\item
Sea $A \in \mathcal{M}_{n\times m}(\mathbb{K})$  donde $\mathbb{K} = \mathbb{R}  \vee \mathbb{C}$, donde  $A=(a_{ij}).$

\item
Se define la matriz transpuesta de A, como la matriz $A^T \in \mathcal{M}_{m\times n}(\mathbb{K})$ obtenida al intercambiar filas por columnas de la matriz A.

\item
Esto es, la fila $i$-ésima de $A$, corresponde a columna $i$-ésima de $A^T$.

\item
Se tiene que $A^T=(a_{ij}^T)$, entonces $a_{ij}^T= a_{ji}$.

\end{itemize}


{Ejemplo}


Sea $A=
\left(
\begin{array}{cc}
2 &  1  \\
0 & -3 \\
1 & -1
\end{array}
\right)
$

Entonces,
 $A^T=
\left(
\begin{array}{ccc}
2 &  0 & 1  \\
1 & - 3  & -1
\end{array}
\right)
$

{Proposición}

Sea $\alpha \in \R, k \in \N, A, B$ 
matrices,  se asume que tienen el orden  apropiado para que cada operación tenga sentido.

\begin{enumerate}
\item
$(A^T)^T = A$

\item
$(A+B)^T = A^T  + B^T $

\item
$(\alpha \cdot A)^T = \alpha \cdot A^T $

\item
$(A \cdot B)^T = B^T \cdot  A^T   $

\item
$(A ^k)^T = (A ^T)^k   $.
\end{enumerate}

{Demostración de algunas Propiedades}
Trabajaremos  en vivo y en directo esto.


{Definición}

Sea $A$ una matriz cuadrada de orden $n$.

\begin{itemize}
\item
$A$ es simétrica si $A^T = A   $

\item
$A$ es antisimétrica si $A^T = - A   $

\end{itemize}


{Ejemplo}

Sea $A=
\left(
\begin{array}{ccc}
2 &  1  &  7 \\
-1 &  0 &  -3 \\
0  &  1 & -1
\end{array}
\right)
$

no es simétrica, pues:
$A^T=
\left(
\begin{array}{ccc}
2  &  -1  &  0 \\
1  &  0   &  1 \\
7  &  -3  & -1
\end{array}
\right)
\neq  A$


Sea $B=
\left(
\begin{array}{ccc}
-3 &  1  &  0 \\
 1 &  5  &  -8 \\
0  &  -8 &  -1
\end{array}
\right)
$

es simétrica, pues:

$B^T=
\left(
\begin{array}{ccc}
-3 &  1  &  0 \\
 1 &  5  &  -8 \\
0  &  -8 &  -1
\end{array}
\right)
= B$

¿Qué característica debería tener una matriz para ser 
antisimétrica?


{Ejemplo}

Sea $A=
\left(
\begin{array}{ccc}
?     &   -4   &  2  \\
4   & ?         &  -3  \\
-2  &  3    &    ? 
\end{array}
\right)
$

entonces, $A=
\left(
\begin{array}{ccc}
0  &   -4   &  2 \\
4   &  0    &  -3\\
-2  &  3    &    0
\end{array}
\right)
$

\begin{alertblock}{Entonces...}
Toda matriz antisimétrica tiene en su diagonal principal, solo ceros.
\end{alertblock}


{Demostración}
$A^T = -A \Rightarrow A^T + A = 0   \Rightarrow a_{ii} + a_{ii} = 0, \forall i \Rightarrow a_{ii} = 0, \forall i$ 


{Proposición}

\begin{itemize}
\item
Sean $A$ y $B$ matrices simétricas del mismo orden:

  \begin{enumerate}
      \item
         A + B es simétrica
      \item
        Si $\alpha \in R$ entonces  $\alpha \cdot A$ es simétrica
  \end{enumerate}

\item
 Si A es una matriz cuadrada entonces:

\begin{enumerate}
\item
$A +A^T$ es simétrica

\item
$A \cdot A^T$  y $A^T \cdot A$ son matrices simétricas.

\item
$A -A^T$ es antisimétrica
\end{enumerate}
\end{itemize}


{Demostración de algunas Propiedades}

Queda como un ejercicio  propuesto...


{Proposición}

Toda matriz $A$ cuadrada se puede descomponer en una parte simétrica y otra antisimétrica.

Demostración

Ejercicio Propuesto...

Sugerencia: usar proposición anterior.


{Ejercicios Propuestos}
1.- Sea 

 $ A= \begin{bmatrix}
2 &  1 & -2 \\
-3 &  0 &  4
\end{bmatrix}$
, 
 $ B= \begin{bmatrix}
 0 &   1\\
-3 &  2  \\
 8 &  1
\end{bmatrix}$
, 
 $ C= \begin{bmatrix}
2   &  -2   & 4   \\
5   &  -6  &  7\\
2   &   1   & -3
\end{bmatrix}$
,
$ D= \begin{bmatrix}
2 &   1\\
-3 &  2  \\
 3  &  4
\end{bmatrix}$
y 
$ E= \begin{bmatrix}
2   &  -2   & 1   \\
3   &  5    &  6\\
-5  &   1   & 4
\end{bmatrix}$

Calcular, si es posible:  
$A^T D$, $(AB)^T$, $3A+2B^T$, $B^T+A^T$ y $(C+E)^T$.

2.- Resolver la ecuación matricial para $X \in \M_{2\times 2}(\R)$: $2X +A^T=A+B^2$, donde:
 $A= \begin{bmatrix}
2 &   1\\
1 &  0
\end{bmatrix}$
;
 $B= \begin{bmatrix}
3 &   1\\
1 &  2
\end{bmatrix}$.

3.- Sean
$ A= \begin{bmatrix}
1 & 2 & 0\\
0 & -1 & 0\\
1 & 0 & 1
\end{bmatrix}$
y
$ B = \begin{bmatrix}
0 & -1 & 1\\
1 & 2 & 0\\
0 & 1 & 0
\end{bmatrix}$,
determinar $ X \in \M_{3\times 3}$ tal que 
$(A^2 + I_{3\times 3} )^t +BAX = A + B^t$

4.- Sea
$ A= \begin{bmatrix}
1 & 1 & 1\\
1 & 1 & 1\\
1 & 1 & 1
\end{bmatrix}$,
deduzca una fórmula para $A^n$.

5.- Sea
$B= \begin{bmatrix}
p & 1 &  0  \\
0 & p & 1 \\
0 & 0 &  p
\end{bmatrix}$, 
demostrar que $\forall n \in \N, B^n=
\begin{bmatrix}
p^n   &    np^{n-1}  &  \frac{n(n-1)}{2}p^{n-2} \\
0        & p^n             &    n p^{n-1} \\
0        & 0                  &     p^{n}
\end{bmatrix}$

6.- Sea
$S= \begin{bmatrix}
0 & 1 & 0 &  0   & \cdots & 0  \\
0 & 0 & 1 &  0   & \cdots & 0 \\
0 & 0 & 0 & 1   & \cdots & 0 \\
\vdots & \vdots & \vdots &  \ddots   & \cdots & \vdots \\
0 & 0 &  &  0   & \cdots & 1 \\
0 & 0 &  &  0   & \cdots & 0
\end{bmatrix}$

Se pide:

a) Determinar $S^n, \forall n \in \N$.

b) Si $A$ es una matriz de orden $n \times n$ encontrar una regla para calcular $S \cdot A$ y $A \cdot  S$.


7.- Hallar una matriz $A$ de orden 2 tal que: 

a) $A^2= (0)$ (unipotente)

b) $A^2= I_2$ (involutiva)

c) $A^2= A$ (idempotente)

d) $A^k= 0$ (nilpotente de orden $k$, donde $k$ es el menor entero que verifica esto)

e) $ A^2=2 \cdot A$

8.- Si $A$ es una matriz involutiva, demostrar que:

a) $\frac{1}{2} (I_n + A)$ y  $\frac{1}{2} (I_n - A)$ son matrices idempotentes.

b) $\frac{1}{2} (I_n + A) \cdot \frac{1}{2} (I_n - A) = 0$ 

9.- Sean $A$ y $B$ matrices simétricas determine si las siguientes matrices son o no simétricas:

a) $A^2 - B^2$

b)$  (A+B) \cdot (A-B)$

c) $ A  \cdot B  \cdot A$

d)$ A  \cdot B  \cdot A  \cdot B$.


10.- Sea
$A= \begin{bmatrix}
1 & 2 &  0  \\
0 & -1 & 0 \\
1 & 0 &  1
\end{bmatrix}$
y
$B= \begin{bmatrix}
0 & -1 &  1  \\
1 & 2 & 0 \\
0 & 1 &  0
\end{bmatrix}$

Obtener la matriz $X \in \M_{3 \times 3}$ si 
$(A^2+I_3)^T +BAX = (A+B^T)^T$

11.- Sea $A$ una matriz de orden $n\times 1$ tal que 
$A^t A= [ 1 ]_{1\times 1} $ y considere una matriz $B$ de orden
 $n\times n$ definida por $B = I_n − 2AA^t$
 
a) Demuestre que $B$ es una matriz simétrica.

b) Utilizando la hipótesis que $A^t A= [ 1 ]$, demuestre que 
$B^2 = I_n$.

c) >Existe la matriz $B^{-1}$?. Justifique claramente su respuesta.


12.- Se define:
$A= (a_{ij})_{2\times 2}$  tal que
 $\forall i, j \in \N : a_{ij} = i + j − 1$
 
$C = (c_{ij})_{2\times 3}$ tal que
  $\forall i, j \in \N : a_{ij} = i - j$
  
Encontrar una matriz $B = (b_{ij})_{3\times 2}$
 que satisfaga la igualdad $AB^t = C$.


{Sistemas de Ecuaciones Lineales}

Usaremos un ejemplo para mostrar la forma matricial:

\begin{eqnarray*}
x+ y +z  &=& 1\\
2x +2y  + z &=&  2\\
x +y  &=&  1
\end{eqnarray*}

Escribiremos la forma matricial de este sistema como:

 $$ \begin{bmatrix}
1 &  1 & 1 \\
2 &  2 & 1 \\
1 & 1 &  0
\end{bmatrix}
\cdot
 \begin{bmatrix}
x \\
y \\
z
\end{bmatrix}
=
 \begin{bmatrix}
1 \\
2 \\
1
\end{bmatrix}
$$


{Sistemas de Ecuaciones Lineales}

En general, consideramos el sistema ecuaciones de $n$ ecuaciones y de $m$ incógnitas:

\begin{eqnarray*}
a_{11}x_1   +   a_{12}x_2   +   \cdots   +   a_{1m}x_m   &=&   b_1\\
a_{21}x_1   +   a_{22}x_2   +   \cdots   +   a_{2m}x_m   &=&   b_2\\
\vdots\\
a_{n1}x_1   +   a_{n2}x_2   +   \cdots   +   a_{nm}x_m   &=&   b_n
\end{eqnarray*}

Escribiremos la forma matricial de este sistema como:

 $$ \begin{bmatrix}
a_{11}    &  a_{12}  & \cdots &  a_{1m}\\
a_{21}    &  a_{22}  & \cdots &  a_{2m}\\
\vdots\\
a_{n1}    &  a_{n2}  & \cdots &  a_{nm}
\end{bmatrix}
\cdot
 \begin{bmatrix}
x_1 \\
x_2\\
\vdots\\
x_m
\end{bmatrix}
=
 \begin{bmatrix}
b_1 \\
b_2 \\
\vdots\\
b_n \\
\end{bmatrix}
$$


Así, podemos escribir: 

$$AX=B$$

donde:

 $A$ se llama matriz de coeficientes,  $A \in \M_{n\times m}(\R)$.

 $X$ se llama matriz de incógnitas,  $X \in \M_{m\times 1}(\R)$.
 
 $B$ se llama matriz de términos independientes,  $B \in \M_{n\times 1}(\R)$.

Definición: 

Consideremos el sistema de  $n$ ecuaciones y de $m$ incognitas:
$$AX=B$$.

\begin{itemize}
\item
$X_0$ es una solución si $X_0 \in \M_{m\times 1}(\R)$, tal que $AX_0=B$.

\item
Un sistema se llama compatible o consistente si tiene al menos una solución.

\item
Un sistema se llama incompatible o inconsistente si no tiene solución.
\end{itemize}


{Operaciones Elementales}
En una matriz podemos realizar 3 tipos de Operaciones Elementales:

\begin{itemize}
\item
Intercambiar (permutar) dos de sus filas.
\item
Multiplicar una fila (es decir cada coeficiente de la fila) por una constante distinta de cero.
\item
Sumar el multiplo de una fila a otra fila.
\end{itemize}


{Ejemplo}

Sea  $ A= \begin{bmatrix}
1 &  3 & -2 \\
-1 &  5 & 0 \\
2 &  0 &  -1
\end{bmatrix}$


\begin{itemize}
\item
Intercambiar fila 1 y 3.

$ A= \begin{bmatrix}
1 &  3 & -2 \\
-1 &  5 & 0 \\
2 &  0 &  -1
\end{bmatrix}
\sim
\begin{bmatrix}
2 &  0 &  -1\\
-1 &  5 & 0 \\
1 &  3 & -2 
\end{bmatrix}
$

\item
Multiplicar la fila 2 por 7.

$ A= \begin{bmatrix}
1 &  3 & -2 \\
-1 &  5 & 0 \\
2 &  0 &  -1
\end{bmatrix}
\sim
\begin{bmatrix}
1 &  3 & -2 \\
-7 &  35 & 0 \\
2 &  0 &  -1
\end{bmatrix}$

\item
Multiplicar la fila 2 por -1 y lo sumamos a la fila 3.

$ A= \begin{bmatrix}
1 &  3 & -2 \\
-1 &  5 & 0 \\
2 &  0 &  -1
\end{bmatrix}
\sim
\begin{bmatrix}
1 &  3 & -2 \\
-1 &  5 & 0 \\
3 &  -5 &  -1
\end{bmatrix}
$
\end{itemize}


{Observación}
\begin{itemize}
\item
Obviamente, el resultado de cada una de estas operaciones no es igual a la matriz  original.
\item
Necesitaremos una notación especial para estas operaciones.

\end{itemize}


{Matrices Elementales}
\begin{itemize}
\item
Una matriz elemental es una matriz que resulta al efectuar solo una operación elemental sobre la matriz identidad $I_n$.

\item
Dado que existen 3 tipos de operaciones elementales, existen 3 tipos de matrices elementales.

\item
$E_{ij}$: matriz elemental obtenida al intercambiar en $I_n$ la fila $i$ con la fila $j$.

\item
$E_{i}(\lambda)$: matriz elemental obtenida al multiplicar por $\lambda$ la fila $i$  de la matriz  $I_n$.

\item
$E_{ij}(\lambda)$: matriz elemental obtenida sumando a la fila $i$,  la fila $j$  de la matriz $I_n$ multiplicada por $\lambda$ (el resultado se guarda en la fila $i$).

\end{itemize}


{Ejemplo}
Matrices elementales de orden 4.
$I_4=\begin{bmatrix}
1 &  0 & 0 & 0\\
0 &  1 & 0 & 0\\
0 &  0 & 1 & 0\\
0 &  0 & 0 & 1
\end{bmatrix}
$


$E_{24}=\begin{bmatrix}
1 &  0 & 0 & 0\\
0 &  0 & 0 & 1\\
0 &  0 & 1 & 0\\
0 &  1 & 0 & 0
\end{bmatrix}
$

$E_{2}(-9)
=\begin{bmatrix}
1 &  0 & 0 & 0\\
0 &  -9 & 0 & 0\\
0 &  0 & 1 & 0\\
0 &  0 & 0 & 1
\end{bmatrix}
$


$E_{41}(-2)=\begin{bmatrix}
1 &  0 & 0 & 0\\
0 &  1 & 0 & 0\\
0 &  0 & 1 & 0\\
-2 &  0 & 0 & 1
\end{bmatrix}
$


{Ejemplo}
Considere ahora la matriz:
$A=\begin{bmatrix}
-1 &  1 & 0 & 2\\
-2 &  3 & -1 & 1\\
 1 & 0  & 0 & 3\\
0 &  1 & 2 & -7
\end{bmatrix}
$

Realizemos  la operación $E_{24}A$.

$=\begin{bmatrix}
1 &  0 & 0 & 0\\
0 &  0 & 0 & 1\\
0 &  0 & 1 & 0\\
0 &  1 & 0 & 0
\end{bmatrix}
\cdot
\begin{bmatrix}
-1 &  1 & 0 & 2\\
-2 &  3 & -1 & 1\\
 1 & 0  & 0 & 3\\
0 &  1 & 2 & -7
\end{bmatrix}
$

$=\begin{bmatrix}
-1 &  1 & 0 & 2\\
0 &  1 & 2 & -7\\
 1 & 0  & 0 & 3\\
-2 &  3 & -1 & 1
\end{bmatrix}$

Notar que es lo mismo que haber efectuado sobre la matriz $A$ la operación elemental de intercambiar la fila 2 con la fila 4.


{Ejercicio}

Efectuar las operaciones:


$E_{2}(-9)\cdot A$


$E_{41}(-2)\cdot A$



\begin{block}{Teorema:}
Sea E la matriz elemental obtenida al efectuar una operación
elemental por filas sobre $I_n$, si la misma operación se realiza sobre
una matriz  $A$ de orden $n\times m$, el resultado es el mismo que el producto $E\cdot A$.
\end{block}


\begin{block}{Observación:}
\begin{itemize}
\item
Notar que la matriz $E$ multiplica por la izquierda a la matriz  $A$.

\item
Si se multiplica por la derecha, esto es $ A\cdot E$, se obtienen las operaciones
elementales por columnas.

\end{itemize}
\end{block}


{Matrices  Equivalentes por Filas}

\begin{itemize}
\item
Las matrices $A$ y $B$ son equivalentes por filas, si existe una sucesión de  operaciones elementales por filas que convierten $A$ en la matriz  $B$.

\item
Esto es, $\exists E_1, E_2, E_3, \cdots,E_k$ operaciones elementales tales que
$E_k \cdots E_3 \cdot E_2 \cdot  E_1 \cdot A= B$.

\item
Notación:
$A\sim B$.
\end{itemize}


{Matriz Escalonada}

\begin{block}{Definición}
Una matriz tiene FORMA ESCALONADA POR FILAS si:
\begin{itemize}
\item
Cualquier fila que se componga enteramente de ceros se ubica en la parte inferior de la matriz.

\item
En cada fila distinta de cero, la primera entrada o coeficiente no nulo (de izquierda a derecha)- denominado pivote- se localiza a la derecha del primer elemento no nulo de la fila anterior.

\end{itemize}

\end{block}


\begin{block}{Definición}

Si además se cumple:

\begin{itemize}
\item
Sus pivotes son todos iguales a 1
\item
En cada fila el pivote es el único elemento no nulo de su columna
\end{itemize}

se dice que la matriz es ESCALONADA REDUCIDA.
\end{block}


{Ejemplos}

\begin{itemize}
\item
$A=\begin{bmatrix}
-1 &  1 & 0 &  2\\
0 &  0 & 2 & -7\\
0 &  0 & 0 &  1
\end{bmatrix}$
es una matriz escalonada.

\item
$B=\begin{bmatrix}
-1 &  6 & 4 &  2\\
0 &  0 & 3 & -1\\
0 &  -2 & 0 &  5
\end{bmatrix}$
no es una matriz escalonada.

\item
Si $A$ es una matriz cuadrada, entonces coincide la definición  con matriz triangular superior.
\end{itemize}


{Ejemplos}
Usaremos los siguientes ejemplos para explicar el método que formalmente definiremos después.

\begin{itemize}
\item
Sea
$A=\begin{bmatrix}
2  &  5  & 6  &  4\\
-1 &  2  & 1  &  0\\
3  &  -1 & 0  &  -5\\
0  &   2 & 3  &  4
\end{bmatrix}$

Realizamos la operación $E_{12}$, solo para evitar operaciones complicadas.

$A
\begin{array}{c}
 E_{12}\\
\sim
\end{array}
\begin{bmatrix}
-1 &  2  & 1  &  0\\
2  &  5  & 6  &  4\\
3  &  -1 & 0  &  -5\\
0  &   2 & 3  &  4
\end{bmatrix}
$
\end{itemize}


{Ejemplos}
{Se inicia en la primera columna.}


Realizar operaciones elementales de modo que todos los coeficientes bajo $a_{11}$ sean ceros.


 $a_{11} = -1$ se llama PIVOTE.

$\begin{bmatrix}
-1 &  2  & 1  &  0\\
2  &  5  & 6  &  4\\
3  &  -1 & 0  &  -5\\
0  &   2 & 3  &  4
\end{bmatrix}$

$\begin{array}{c}
 E_{21}(2)\\
 E_{31}(3)\\
\sim
\end{array}$

$\begin{bmatrix}
-1 &  2  & 1  &  0\\
0  &  9  & 8  &  4\\
0  &  5 & 3  &  -5\\
0  &   2 & 3  &  4
\end{bmatrix}
$

{Terminamos con la columna 1.}

{Pasamos a la columna 2.}

Ahora el PIVOTE es $a_{22}=9$

$$
\begin{bmatrix}
-1 &  2  & 1  &  0\\
0  &  9  & 8  &  4\\
0  &  5 & 3  &  -5\\
0  &   2 & 3  &  4
\end{bmatrix}
\begin{array}{c}
 E_{32}(\frac{-5}{9})\\
 E_{42}(\frac{-2}{9})\\
\sim
\end{array}
\begin{bmatrix}
-1 &  2  & 1  &  0\\
0  &  9  & 8  &  4\\
0  &  0 & \frac{-13}{9}  &  \frac{-65}{9}\\
0  &  0 & \frac{-11}{9}  &  \frac{-28}{9}
\end{bmatrix}
$$

Terminamos con la columna 2.


{Ejemplos}
\shadowbox{Pasamos a la columna 3.}

Solo para escribir mas simple la matriz:


$$
\begin{bmatrix}
-1 &  2  & 1  &  0\\
0  &  9  & 8  &  4\\
0  &  0 & \frac{-13}{9}  &  \frac{-65}{9}\\
0  &  0 & \frac{-11}{9}  &  \frac{-28}{9}
\end{bmatrix}
\begin{array}{c}
 E_{3}(9)\\
 E_{4}(9)\\
\sim
\end{array}$$

$$\begin{bmatrix}
-1 &  2  & 1  &  0\\
0  &  9  & 8  &  4\\
0  &  0 & -13  &  -65\\
0  &  0 & -11  &  -28
\end{bmatrix}
$$


Ahora el PIVOTE es $a_{33}=-13$

$$\begin{array}{c}
 E_{43}(\frac{-11}{13})\\
\sim
\end{array}$$

$$\begin{bmatrix}
-1 &  2  & 1  &  0\\
0  &  9  & 8  &  4\\
0  &  0 & -13  &  -65\\
0  &  0 &  0  &  83
\end{bmatrix}
$$


Podemos obtener mas ceros en la última fila?


{Ejercicios}
Repetir el procedimiento en las siguientes matrices:


$B=
\begin{bmatrix}
2 & 0    & -1  &  0\\
1  &  1  & 1   &  2\\
3  &   2 & 1   &  1
\end{bmatrix}
$

$C=
\begin{bmatrix}
1   & 1    &  0\\
1   &  2   &  1\\
 -1 &   0  & 1
\end{bmatrix}
$

$D=
\begin{bmatrix}
 -1  &  1\\
-2  &  2\\
3   & -3 
\end{bmatrix}
$


{Rango}

\begin{block}{Definición}
Sea $A$ una matriz cualquiera, se denomina rango de la matriz $A$ al número de filas no nulas de la matriz equivalente por filas escalonada de $A$.

Notación:

Rango($A$), $\rho(A)$, R(A).
\end{block}

\begin{block}{Observación}

1) Debe ser la matriz equivalente por filas escalonada con mayor cantidad de ceros posibles.

2) Este número es único para cada matriz y es una propiedad inherente a ella.

\end{block}


Ejemplos: revisar las matrices escalonadas del ejemplo anterior.



{Ejercicio Propuesto}

Hallar el rango de las matrices:

$A_n=
\begin{bmatrix}
n+1  &  1       &  1\\
1       &  n+1  & 1\\
1       &   1     & n+1 
\end{bmatrix}
$

$B_n=
\begin{bmatrix}
n+1  &  1       &  n\\
1       &  n+1  & 1\\
0       &   0     & n 
\end{bmatrix}
$

{Pivotes}
Recordar:
\begin{block}{Definición}
En una matriz escalonada los pivotes son tales que:

\begin{itemize}
\item
Todos los elementos debajo del pivote son ceros.

\item
el primer elemento no nulo de cada fila (pivote) está a la derecha del primer elemento no nulo de la fila anterior.
\end{itemize}
\end{block}


\begin{block}{Definición}
Si además:

\begin{itemize}
\item
Los  pivotes son los únicos elementos no nulos de cada columna.

\item
Los pivotes son iguales a 1.
\end{itemize}

La matriz se llama escalonada reducida.
\end{block}


{Algoritmo}

\begin{block}{Definición}
Un algoritmo es una secuencia finita de operaciones realizables, no ambiguas, que se ejecutan en un tiempo finito.
\end{block}


Ejemplo...

\begin{block}{Teorema}
Sea $A$ una matriz cualquiera, entonces existen matrices $E$ y $U$ tales que $E\cdot A=U$, donde U es una matriz escalonada.
\end{block}

La demostración de este teorema da origen al Método de Gauss.


{Sistema Homogéneo}

\begin{block}{Definición}
El sistema ecuaciones de $n$ ecuaciones y de $m$ incógnitas:
$AX=0$ se llama Sistema Homogéneo.
\end{block}

\begin{block}{Propiedades de los Sistemas Homogéneos}
\begin{enumerate}
\item
Siempre son compatibles.

\item
$C \in \M_{n\times m}(\R)$ tal que $A\sim C$, entonces 
$AX=0$  y $CX=0$ tienen las mismas soluciones.
\end{enumerate}
\end{block}


{Sistema no Homogéneo}
\begin{block}{Definición}
El sistema ecuaciones de $n$ ecuaciones y de $m$ incógnitas:
$AX=B$ donde $B \neq 0$ se llama Sistema no Homogéneo.
\end{block}

\begin{block}{Definición}
Para $AX=B$ se define la matriz ampliada: $(A,B)= (A|B)=(A\vdots B)$, de orden $n\times (m+1)$.
\end{block}


{Sistema no Homogéneo}
\begin{block}{Propiedades}
\begin{enumerate}
\item
Para el sistema ecuaciones de $n$ ecuaciones y de $m$ incógnitas:

Las soluciones de $AX=B$ son las mismas que las del sistema $CX=D$, donde $C = E\cdot A$ y $D= E\cdot B$.

\item
El sistema  $AX=B$ es compatible si $\rho(A) = \rho(A,B)$.

\item
Si el sistema  $AX=B$ es compatible:

a) $\rho(A) = \rho(A,B) = m$ (número de incognitas), entonces el sistema tiene solución única.

b) $\rho(A) = \rho(A,B) < m$, entonces el sistema tiene infinitas soluciones.

\end{enumerate}
\end{block}


{Eliminación Gaussiana}
\begin{itemize}
\item
Construir: $(A,B)$
\item
Aplicar el Algoritmo Gauss (matriz escalonada).
\item
Sustitución Regresiva.
\end{itemize}


{Ejercicios Propuestos}

Resolver (indicando el tipo de solución):

\begin{enumerate}
\item
\begin{eqnarray*}
2x  +  y   +  z   &=& 1\\
4x  +  y           &=&  -2\\
-2x + 2y  + z  &=&  7
\end{eqnarray*}


\item
\begin{eqnarray*}
2x  + y   - z   +  t  &=& 0\\
x    +2y   + z -   t  &=&  0\\
3x -  y      -2t        &=&  0
\end{eqnarray*}

\item
\begin{eqnarray*}
x    -  2y   +   2z    &=& 3\\
3x -    5y +   4z    &=&  1\\
2x   -  3y    +2z    &=&  1
\end{eqnarray*}

\end{enumerate}


{Eliminación Gauss - Jordan}
\begin{itemize}
\item
Construir: $(A,B)$
\item
Algoritmo Gauss (matriz escalonada reducida).
\item
La solución es inmediata.
\end{itemize}


{Ejercicios Propuestos}

1.- Resolver (indicando el tipo de solución):

\begin{enumerate}
\item
\begin{eqnarray*}
x- 3y +z  &=& -2\\
2x +y  - z &=&  6\\
x +2y  +2z&=&  2
\end{eqnarray*}

\item
\begin{eqnarray*}
x   + y      +  z  &=& 1\\
2x +  2y  + z   &=&  2\\
x   +  y             &=&  1
\end{eqnarray*}

\end{enumerate}

2.- Hallar $k$ de modo que el sistema:

\begin{eqnarray*}
kx  + y +z  &=& 5\\
3x  +2y  +kz &=&  18-5k\\
y  +2z&=&  2
\end{eqnarray*}

a) no tenga solución.

b) tenga infinitas soluciones.

3.- Hallar $a$ y $b$, de modo que el sistema, tenga:

\begin{eqnarray*}
2ax - by + 2z &=& 1\\
x - 2aby + 2z &=& 2b\\
ax + (2a^2 -1)by + 2(1 - a)z &=& 1 - 2ab
\end{eqnarray*}

a) Solución única.

b) Infinitas soluciones.

4.- Determine el rango de la matriz
$A= \begin{bmatrix}
a&b&0 \\
a&0&b\\
0&a&b
\end{bmatrix}
$
suponga que $a>0, b>0$

5.- Resuelva el sistema de ecuaciones
\begin{eqnarray*}
a(x - 1) + b(y - 1) &=& a + b\\
a(x - 1) + b(z - 1) &=& 0\\
a(y - 1) + b(z - 1) &=& 0
\end{eqnarray*}

6.- Considere el sistema lineal definido en $\R$:
\begin{eqnarray*}-
x +  ay + 3z &=& 2\\
x + (2a -1)y + 2z &=& 2\\
x + ay + (a+4)z &=& 2a+4
\end{eqnarray*}

Determine todos los valores $a \in \R$ para que el sistema:

a) tenga solución única.

b) no tenga solución.

c)  tenga infinitas soluciones.


7.- Un grupo de ni\~nos decide comprar bebidas, papas fritas y una torta.
 Ellos saben que si compran 2 bebidas,1 torta y n paquetes de papas
 fritas gastarán 5.550 pesos. Si compraran 3 bebidas, 1 torta y 7
 paquete de papas fritas, gastarán 5.850 pesos, pero si compraran n
 bebidas, 1 torta y 10 paquetes de papas fritas gastarían
6.150 pesos. Con estos datos >existe solución para el precio de 1
 bebida, 1 torta y 1 paquete de papas fritas?
 
8.- Encuentre coeficientes reales $a, b, c$ y $d$ tal que la gráfica
  de la función $f(x)= ax^3 + bx^2+ cx + d$;
pase por los puntos $(1,2)$ ; $(-1,6)$ ; $(2,3)$ ; $(-2,1)$.

9.- Determine condiciones sobre los números reales $a$ y $b$,
 de modo que el sistema:
 
 \begin{eqnarray*}
(a^2+1)x   +  2y + 3z     &=& 0\\
-2(a^2+1)x +  2y + 3z     &=& 6b\\
           -  2y + (a-4)z &=& b+1
\end{eqnarray*}

(i) Tenga infinitas soluciones.

(ii) Tenga una solución única.

(iii) No tenga soluciones.


{Matriz Inversa}

\begin{block}{Definición}
Sea $A$ una matriz cuadrada de orden $n$, se dice que $A$ es invertible  si existe una matriz  cuadrada de orden $n$, que denotaremos $A^{-1}$ tal que:

$$A \cdot A^{-1} = A^{-1}  \cdot A = I_n$$

\end{block}


\begin{block}{Observación}
\begin{enumerate}
\item
También se denomina regular, no singular o inversible.

\item
Si una matriz inversa existe, entonces es única.
(demostrar)
\end{enumerate}
\end{block}


{Proposición}

Sean $A,B \in \M(n,\R)$ matrices no singulares:

\begin{enumerate}
\item
$(AB)^{-1} = B^{-1}A^{-1}$.
\item
$(A^{-1} )^{-1} = A$.
\item
$(A^T )^{-1} = (A^{-1} )^T$.
\item
$(\alpha A)^{-1}  = \frac{1}{\alpha} A^{-1}, \alpha  \neq 0 $.
\item
$(A^k )^{-1} = (A^{-1} )^k, \forall k \in \N$.
\end{enumerate}

(demostrar)


{Observación}


\begin{enumerate}
\item
En general, podemos definir:

$A^k = \left\{ \begin{array}{rcl}
A\cdot A \cdot A \cdots A &, k\in \N\\
A^0= I_n &, k=0 \\
(A^{-1} )^m, & m=-k, k<0
\end{array}\right.$

\item
Las matrices elementales son todas invertibles:
\begin{enumerate}
\item
$E_{ij}^{-1} = E_{ji}$

\item
$E_{i}^{-1} = E_{i}$

\item
$E_{ij}^{-1} (\lambda)= E_{ij}(-\lambda)$
\end{enumerate}
\end{enumerate}


{Ejercicios Propuestos}

\begin{enumerate}
\item
Sean  $A,B \in \M(n,\R)$ matrices no singulares tal que $A$ es simétrica y $B^T=2B^{-1}$. Encuentre $X \in \M(n,\R)$ en la expresión:

$$XA+(AB)^{T} = (A^{-1}B)^{-1}$$

\item
Resolver la ecuación matricial: $(AX^{-1}B)^T= AB$, donde:

$A= \begin{bmatrix}
1&1& 1 \\
0&1&2\\
3&1&0
\end{bmatrix}
$
y
$B= \begin{bmatrix}
0&0& 1 \\
1&2&0\\
1&1&0
\end{bmatrix}
$
(son matrices invertibles).
\item
Sean $A,B$ matrices regulares y simétricas. 

a) $AB^T$ es simétrica e invertible.

b) $(A^{-1})^T = (A^T)^{-1}$.

c) $(A+B)^{-1} = A^{-1}  +  B^{-1}.$

¿Cuáles son verdaderas?

\end{enumerate}


{Teoremas}
\begin{block}{Teorema 1}

Una matriz cuadrada $A$  de orden $n$ es invertible si y solo si $\rho(A)=n$.
\end{block}

\begin{block}{Teorema 2}

$\exists A^{-1} \Leftrightarrow A \sim I_n$.
\end{block}

\begin{block}{Teorema 3}

Sea $A$ una matriz de orden $n$ invertible. Si una sucesión de operaciones elementales por filas $E_1, E_2, \cdots, E_k$ transforma la la matriz $A$ en la identidad $ I_n$.

Entonces la misma sucesión de operaciones elementales convierte la matriz  $ I_n$ en $A^{-1}$.
\end{block}


{Método G-J}
\begin{block}{Método de Gauss Jordan: Inversa de una Matriz}

Sea $A$ una matriz invertible.

\begin{enumerate}
\item
Se construye la matriz aumentada: $(A,I_n)$.
\item
$(A,I_n)
\begin{array}{c}
\mbox{operaciones}\\
\sim\\
\mbox{elementales}
\end{array}
 (I_n,E) \Rightarrow E=A^{-1}$
\end{enumerate}
\end{block}


{Ejemplos}
Calcular la inversa, en caso de existir:

\begin{enumerate}
\item
$A= \begin{bmatrix}
2 &  -1 & 1 \\
1 &  -2 & 0\\
0 &  1 &  2
\end{bmatrix}
$

\item
$B= \begin{bmatrix}
1 &  -1 & 1 \\
2 &  2 & 2\\
0 &  0 &  1
\end{bmatrix}
$

\item
$C= \begin{bmatrix}
1 &  0 & 0  & 0 \\
2 &  1 & 0  & 0\\
3 &  3 &1  & 0\\
4 &  4 &4  & 1
\end{bmatrix}
$
\end{enumerate}


{Determinante}

\begin{block}{Definición}
Sea $A \in \M(n,\R)$, el determinante de $A$ es una función 
$det: \M(n,\R) \longrightarrow \R$, que a cada matriz $A$ le asocia un número real $det(A) = |A|$.
\end{block}

Aplicación:

\begin{itemize}
\item
Criterio para analizar si  una matriz tiene inversa.
\item
Calcular inversa de una matriz.
\item
Sistemas de Ecuaciones Lineales.
\end{itemize}

\begin{alertblock}{Recursividad}
La definición del determinante es recursiva, previo necesitamos definir...
\end{alertblock}


{Definiciones}

\begin{block}{Definición}
\begin{enumerate}
\item
Una submatriz de $A$ corresponde a una parte de la matriz  $A$.
\item
Sea $A \in \M(n,\R)$ se llama menor de orden $ij$ de $A$ y se denota $M_{ij}$ al determinante de la matriz de orden $n-1$ que resulta de eliminar la fila $i$ y la columna $j$ de $A$.
\item
Se llama cofactor de $ij$ de $A$ al número $c_{ij} = (-1)^{i+j} M_{ij}$.
\end{enumerate}
\end{block}


{Definición Recursiva para Calcular el Determinante}

\begin{block}{Definición}
\begin{enumerate}
\item
Para $n=1:
A=(a_{11})$.

Entonces $det(A) = a_{11}$.

\item
Para $n=2: 
A= \begin{bmatrix}
a_{11} &  a_{12} \\
a_{21} &  a_{22}
\end{bmatrix}
$

Entonces 
$det(A) = a_{11}a_{22} - a_{21}a_{12}$.

\item
Para $n>2:
A=(a_{ij})$

Entonces
$det(A) =
\left\{ \begin{array}{rl}
\sum_{i=1}^n (-1)^{i+j}a_{ij} M_{ij}  = \sum_{i=1}^n a_{ij} c_{ij}
&1\leq j  \leq n \mbox{ con }j \mbox{ fijo.}  \\ 
\sum_{j=1}^n (-1)^{i+j}a_{ij} M_{ij}  =   \sum_{j=1}^n a_{ij} c_{ij} 
& 1\leq i \leq n \mbox{, con } i \mbox{ fijo.}
\end{array}\right.
$
\end{enumerate}
\end{block}


{Observación}

\begin{enumerate}
\item
Podemos calcular el determinante por filas o columnas.
\item
Podemos realizar operaciones por filas (E) o columnas (C).
\end{enumerate}


{Ejemplos}
Calcular el determinante:

\begin{enumerate}
\item
$A=(-6)$
\item
$B= \begin{bmatrix}
3 &  -1 \\
0 &  1
\end{bmatrix}
$
\item
$C= \begin{bmatrix}
1 &  -1 & 1 \\
2 &  2 & 2\\
0 &  0 &  1
\end{bmatrix}
$
\item
$D= \begin{bmatrix}
1 &  0 & 0  & 0 \\
2 &  1 & 0  & 0\\
3 &  3 &1  & 0\\
4 &  4 &4  & 1
\end{bmatrix}
$
\end{enumerate}


{Observación}

\begin{enumerate}
\item
El calculo del determinante de una matriz de orden $n$, se traslada al cálculo de $n$ determinantes de submatrices de orden $n-1$. 
\item
Elegir una columna o fila con la mayor cantidad de ceros, así la cantidad de operaciones será mínima.
\item
El valor del determinante no depende de la fila o columna elegida.
\item
$det(I_n) = 1$.
\end{enumerate}


{Propiedades}

\begin{enumerate}
\item
$det(A) = det(A^T)$.

\item
Si  $A$ posee una fila o columna de ceros, entonces $det(A) =0.$

\item
Si $A=(a_{ij})$ diagonal, triángular superior o inferior, entonces: 
$det(A) = \prod _{i=1}^n a_{ii}$.

\item
$det(\alpha A) = \alpha^n det(A), \alpha \in \R$.

\item
$det(A\cdot B) = det(A) \cdot det(B)$.

\item
Si $A$ tiene dos filas o columnas iguales o proporcionales, entonces $det(A)=0$.

\item
Si se intercambian dos filas o dos columnas en una matriz, entonces su determinante cambia de signo.

$det(E_{ij}A)  = - det(A)$ ; $ det(C_{ij}A) = -  det(A)$.

\item
$det(E_{i}(\alpha )A) = \alpha det(A)$.

$det(C_{i}(\alpha )A) = \alpha det(A)$.

\item
$det(E_{ij}(\alpha )A) =det(A)$.

$det(C_{ij}(\alpha )A) = det(A)$.

\end{enumerate}


{Ejemplos}
Es importante entender cómo utilizar estas propiedades, por tanto, a continuación,  calcularemos algunos determinantes, pero utilizando las propiedades mencionadas anteriormente:

\begin{enumerate}
\item
$A= \begin{bmatrix}
1 &  2 & 3 \\
1 &  3 & 0\\
2 &  4 & 1
\end{bmatrix}
$

\item
$B= \begin{bmatrix}
1 &  3 & 0 \\
-1 &  2 & -4\\
1 &  1 & 2
\end{bmatrix}
$

\item
$C= \begin{bmatrix}
7 &  6 & 8   &5\\
6 &  7 & 10  &  6\\
7 &  8 & 8  &  9\\
8 &  7 & 9  &6
\end{bmatrix}
$

\item
$D= \begin{bmatrix}
x &  1 & 1 \\
1 &  x & 1\\
1 &  1 & x
\end{bmatrix}
$

\end{enumerate}




{Ejercicios Propuestos}

\begin{enumerate}
\item
Resolver:
$\left|
\begin{array}{ccc}
x-a-b &a & b\\
c & x-b-c & b\\
c & a & x-a-c
\end{array}
\right|
=0
$
\item
Calcular:
$\left|
\begin{array}{cccc}
a & a & a   & a\\
a &  b & b  &  b\\
a &  b & c  &  c\\
a &  b & c  &d
\end{array}
\right|
$
y
$\left|
\begin{array}{ccc}
1 & 1 & 1\\
a & b &c\\
a^2 & b^2 & c^2
\end{array}
\right|
$
\item
Mostrar que:
$\left|
\begin{array}{ccc}
y_1+ z_1& z_1+ x_1 & x_1+y_1\\
y_2+ z_2& z_2+ x_2 & x_2+y_2\\
y_3+ z_3& z_3+ x_3 & x_3+y_3
\end{array}
\right|
=
2\left|
\begin{array}{ccc}
x_1 & y_1 & z_1\\
x_2 & y_2 & z_2\\
x_3 & y_3 & z_3
\end{array}
\right|
$

\item
Resolver la ecuación:
$\left|
\begin{array}{cccc}
x & a & b  &c\\
a & x & c  &b\\
b & c & x  &a\\
c & d & a  &x
\end{array}
\right|
$

\item
Sea $n\geq 2$ un entero y sea $D_n =(d_{i,j})$ la matriz cuadrada de 
orden $n$ con las siguientes propiedades:

a) $d_{i,i+5}$, para $i=1, ..., n$;

b) $d_{i+1,i}= 1$, para $i=1, ..., n-1$;

c) $d_{i,i+1}=2,$ para $i=1, ..., n-1$;

d) todos los otros coeficientes $d_{i,j} =  0$.

$D_1$ es la matriz de orden $1\times  1$ cuyo coeficiente es 5.

Obtener una relación entre los determinantes de $D_{n-2}, D_{n-1}$ y $D_n$.

Usar la relación para obtener el determinante de $D_5$.

\item
Sea $A\in \M_4(\R)$ tal que $det(2Adj(A)) = 1024$.
 Calcule $det(\frac{1}{2}A^{-1})$

\end{enumerate}


{Inversa de una matriz}

\begin{block}{Definición}
Sea $A \in \M(n,\R)$, se define la matriz de cofactores $C=(c_{ij})$ 
donde $c_{ij}$ son los cofactores de $A$.

Además, se define la matriz adjunta de $A$, $Adj(A) = C^T$.
\end{block}

Ejemplo...

\begin{block}{Teorema}

 $A \cdot Adj(A) = |A| \cdot  I_n$

 $ Adj(A)  \cdot A= |A| \cdot  I_n$
\end{block}


{Método para calcular la inversa}

\begin{block}{Método}
Del teorema anterior y por unicidad se tiene que 
 $A^{-1} = \frac{1}{ |A|}Adj(A)$

Más aún, 
  $det(A^{-1})  = \frac{1}{ det(A)}$
\end{block}

\begin{block}{Teorema}
A es no singular ssi $det(A) \neq 0$
\end{block}


{Ejemplos}
Hallar la matriz inversa, si es posible, de:

\begin{enumerate}

\item
$A= \left(
\begin{array}{ccc}
2&1&-1\\
0&2&1\\
5&2&-3
\end{array}
\right)$

\item
$B= \left(
\begin{array}{ccc}
1&2&3\\
0&7&-4\\
-1&5&-7
\end{array}
\right)$

\end{enumerate}


{Regla de Cramer}

\begin{block}{Método}
Resolver el sistema de ecuaciones lineales $AX=B$, de n ecuaciones y n incognitas.
 
 $A=(a_{ij})$  ;  $X=(x_i)$  ;   $B= (b_i)$
 
Si  $det(A) \neq 0$ entonces el sistema tiene solución única, dada por:
 
$$x_i = \frac{|A_i|}{|A|}, \forall i=1,n$$
    
donde $A_i$ se obtiene a partir de $A$, al reemplazar la $i$-esima columna por la matriz $B$.

\end{block}


{Ejercicios Propuestos}

Resolver:

$$\begin{bmatrix}
-2  &  3   & -1\\ 
1   & 2    & -1 \\
-2  &  -1  & 1 
\end{bmatrix}
 \cdot
\begin{bmatrix}
x\\ 
y \\
z
\end{bmatrix}
=
\begin{bmatrix}
1\\ 
4 \\
-3
\end{bmatrix}
$$



%%%%%%%%%


Se tiene que:

$A= \begin{bmatrix}
-2  &  3   & -1\\ 
1   & 2    & -1 \\
-2  &  -1  & 1 
\end{bmatrix}
$
;
$
X=
\begin{bmatrix}
x_1\\ 
x_2\\
x_3
\end{bmatrix}
$
y
$
B=
\begin{bmatrix}
1\\ 
4 \\
-3
\end{bmatrix}
$

Notar que $|A| = -2\neq 0$ por tanto el sistema tiene solución única dada por:

$x_1= \frac{\left|
\begin{array}{ccc}
\colorbox{yellow}{1} &  3   & -1\\ 
 \colorbox{yellow}{4}& 2    & -1 \\
 \colorbox{yellow}{-3}&  -1  & 1 
\end{array}
\right|}
{|A|}
=
\frac{-4}{-2}=2$


$x_2= \frac{\left|
\begin{array}{ccc}
-2&  \colorbox{yellow}{1}  & -1\\ 
 1 &  \colorbox{yellow}{4}   & -1 \\
-2 &  \colorbox{yellow}{-3}  & 1 
\end{array}
\right|}
{|A|}
=
\frac{-6}{-2}=3$


$x_3= \frac{\left|
\begin{array}{ccc}
 -2&  3   & \colorbox{yellow}{1}\\ 
1  & 2    &  \colorbox{yellow}{4} \\
-2 &  -1  &   \colorbox{yellow}{-3}
\end{array}
\right|}
{|A|}
=
\frac{-8}{-2}=4$


La solución del sistema es:
$
X=
\begin{bmatrix}
2\\ 
3\\
4
\end{bmatrix}
$

Notar que:

$$\begin{bmatrix}
-2  &  3   & -1\\ 
1   & 2    & -1 \\
-2  &  -1  & 1 
\end{bmatrix}
 \cdot
\begin{bmatrix}
2\\ 
3 \\
4
\end{bmatrix}
=
\begin{bmatrix}
1\\ 
4 \\
-3
\end{bmatrix}
$$


{Una observación para solución de sistemas}

\begin{enumerate}
\item
Los sistemas que aparecen en muchas aplicaciones son de gran tama\~no. 
\item
El tiempo de cálculo del computador necesario para resolver el sistema debe ser lo menor posible. 

\item
Por ejemplo: El Método de Eliminación Gaussiana es mas rápido que la Regla de Cramer (realiza menos operaciones).
\end{enumerate}


{Fin}

\center{Terminamos!!!}


\end{document}
